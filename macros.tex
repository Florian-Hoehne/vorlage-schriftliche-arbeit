\usepackage{xcolor}
\definecolor{coal}{rgb}{0.21, 0.27, 0.31}

% <Command>
% @version	v 1.0
% @author 	Florian-PDV
% @desc		puts an tabulated bullet item inline
% 			does not render page breaks
\newcommand{\tabitem}{~~\llap{{\color{coal}\textbullet}}~~}

% <Command[String]>
% @version	v 1.0
% @author 	Florian-PDV
% @param[1]	a string set of the link to be rendered	
% @desc		creates an url link rendered in bold style
\newcommand{\booklink}[1]{{\color{coal}\textbf{URL:}\url{#1}}}

% <Command[String]>
% @version	v 1.0
% @author 	Florian-PDV
% @param[1]	a string set to be formated
% @desc		creates an bold and cursive text with quotation-marks
%			recommended for bibitem
\newcommand{\book}[1]{\textbf{\textit{\grqq{}#1\grqq{}}}}

% <Command[String]>
% @version	v 1.0
% @author 	Florian-PDV
% @param[1]	a string set to be formatted
% @desc		creates an bold and cursive text with quotation-marks
%			recommended for bibitem
\newcommand{\say}[1]{\grqq{}#1\grqq{}} 

% <Command[String]>
% @version	v 1.0
% @author 	Florian-PDV
% @param[1]	bib reference (fig:#1)
% @desc		creates a reference with title for figures
\newcommand{\Abbildung}[1]{Abbildung~\ref{fig:#1} auf Seite~\pageref{fig:#1}}

% <Command[String]>
% @version	v 1.0
% @author 	Florian-PDV
% @param[1]	bib reference
% @desc		creates a reference with title for the apendix
\newcommand{\Anhang}[1]{Anhang~\ref{#1} - \textit{\nameref{#1}}}

% <Command[String]>
% @version	v 1.0
% @author 	Florian-PDV
% @param[1]	bib reference
% @desc		creates a reference with title for any existing reference
\newcommand{\Punkt}[1]{Punkt~\ref{#1} - \textit{\nameref{#1}}}

% <Command[String]>
% @version	v 1.0
% @author 	Florian-PDV
% @param[1]	bib reference
% @desc		creates a reference with title for the apendix
\newcommand{\Tabelle}[1]{Tabelle~\ref{tab:#1} - \textit{\nameref{tab:#1}}}

% <Command[String]>
% @version	v 1.0
% @author 	Florian-PDV
% @param[1]	bib reference (sec:#1)
% @desc		creates a reference with title for a section
\newcommand{\Abschnitt}[1]{Abschnitt~\ref{sec:#1}}

% <Command[String]>
% @version	v 1.0
% @author 	Florian-PDV
% @param[1]	bib reference (lst:#1)
% @desc		creates a reference with title for a listing
\newcommand{\Listing}[1]{Listing~\ref{lst:#1} - ~\textit{\nameref{lst:#1}}}

% <Command[String]>
% @version	v 1.0
% @author 	Florian-PDV
% @param[1]	bib reference
% @desc		creates a reference with title for the apendix
\newcommand{\Apx}[1]{Abschnitt~\ref{sec_apx:#1} - ~\textit{\nameref{sec_apx:#1}}}

% <Command[String]>
% @version	v 1.0
% @author 	Florian-PDV
% @param[1]	text
% @desc		writes the text in fold font
\newcommand{\bold}[1]{\textbf{#1}}

%##########################################################
% LISTINGS
%##########################################################

% LISTING<Command[Listing]>
% @version	v 1.0
% @author 	Florian-PDV
% @param[1]	filename (folder listings)
% @param[2]	caption of the listing
% @param[3]	label of the listing (lst:#3)
% @desc		renders a listings box
 \newcommand{\node}[3]{\lstinputlisting [language=node, caption={#2}, label={lst:#3}]{listings/#1}}

%##########################################################
% Colors
%##########################################################

% COLOR<Command[Definition]>
% @version	v 1.0
% @author 	Florian-PDV
% @param[1]	hex rgb value for red
% @param[2]	hex rgb value for green
% @param[3]	hex rgb value for blue
% @desc		assigns the primary color 
\newcommand{\primaryColor}[3]{\definecolor{primary}{RGB}{#1,#2,#3}\def\R{#1}\def\G{#2}\def\B{#3}}

% COLOR<Command[Definition]>
% @version	v 1.0
% @author 	Florian-PDV
% @param[1]	hex rgb value for red
% @param[2]	hex rgb value for green
% @param[3]	hex rgb value for blue
% @desc		assigns the secondary color 
\newcommand{\secondaryColor}[3]{\definecolor{secondary}{RGB}{#1,#2,#3}}

%##########################################################
% Helpers
%##########################################################

% HELPER<Command[String]>
% @version	v 1.0
% @author 	Florian-PDV
% @param[1]	heading title (kind of the project)
% @desc		constructs the header for the cover
\newcommand{\HEAD}[1]{\begin{center}{\Huge #1}\linebreak zum Thema \linebreak\end{center}}

% HELPER<Command[String]>
% @version	v 1.0
% @author 	Florian-PDV
% @desc		puts a new line withe 15pt separation
%			used for the cover sections
\newcommand{\fline}{\\[15pt]}

% HELPER<Command[String]>
% @version	v 1.0
% @author 	Florian-PDV
% @param[1]	column heading
% @desc		renders the column header for the cover page
\newcommand{\rowhd}[1]{\textbf{#1:}}

% HELPER<Command[String]>
% @version	v 1.0
% @author 	Florian-PDV
% @param[1] main title of the project
% @desc		assigns the title string to the header variables
%			if the title is longer then 30 characters the right header
%			use the work title instead
\newcommand{\assignTitle}[1]{%
\newcommand{\FullTitle}{#1}
    \newcommand{\Title}{#1}% Zuweisung zu myTitle
    \def\lenght{\newTitle}%
    \def\mystring{#1}%    
    \stringlength[q]{\mystring}% Result is stored in \theresult
    \def\mythresh{30}
    %\ifnum\theresult>\mythresh \def\newTitle{\worktitle} \else \def\newTitle{#1}\fi    
    \ifnum\theresult>\mythresh \newcommand{\newTitle}{\worktitle} \else \newcommand{\newTitle}{#1}\fi    
}
